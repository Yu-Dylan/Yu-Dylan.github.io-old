\pgfmathsetseed{2020} % or any other number: sets the random seed

% This is essentially the only way to reliably use loops, in the \makeatletter environment.
\makeatletter
\def\declarenumlist#1#2#3{%
\expandafter\edef\csname pgfmath@randomlist@#1\endcsname{#3}%
\count@\@ne
\loop
\expandafter\edef
\csname pgfmath@randomlist@#1@\the\count@\endcsname
  {\the\count@}
\ifnum\count@<#3\relax
\advance\count@\@ne
\repeat}

\declarenumlist{hintlist}{1}{\value{hintcounter}}

\def\prunelist#1{%
\expandafter\edef\csname pgfmath@randomlist@#1\endcsname
    {\the\numexpr\csname pgfmath@randomlist@#1\endcsname-1\relax}
\count@\pgfmath@randomtemp
\loop
\expandafter\let
\csname pgfmath@randomlist@#1@\the\count@\expandafter\endcsname
\csname pgfmath@randomlist@#1@\the\numexpr\count@+1\relax\endcsname
\ifnum\count@<\csname pgfmath@randomlist@#1\endcsname\relax
\advance\count@\@ne
\repeat}
\makeatother

\ifdylanhide

\else 
\section{Hints}
\hintprinter
\fi 
